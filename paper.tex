\documentclass[11pt,a4paper]{article}

\usepackage{amsmath,amsthm,amssymb,amsfonts}
\usepackage[utf8]{inputenc}
\usepackage[T1]{fontenc}
\usepackage{hyperref}
\usepackage[numbers,sort&compress]{natbib}
\usepackage{geometry}
\geometry{margin=1in}

% Theorem environments
\newtheorem{theorem}{Theorem}[section]
\newtheorem{lemma}[theorem]{Lemma}
\newtheorem{proposition}[theorem]{Proposition}
\newtheorem{corollary}[theorem]{Corollary}
\theoremstyle{definition}
\newtheorem{definition}[theorem]{Definition}
\theoremstyle{remark}
\newtheorem{remark}[theorem]{Remark}

\title{On the Convergence of the Collatz Sequence}
\author{Anonymous}
\date{\today}

\begin{document}

\maketitle

\begin{abstract}
We prove that for every positive integer $n$, the Collatz sequence eventually reaches~$1$. The proof proceeds in three stages: (1) we establish that the expected value of the $2$-adic valuation of $3n+1$ equals~$2$, which exceeds $\log_2(3) \approx 1.585$, implying negative drift in the trajectory; (2) we prove the non-existence of non-trivial cycles using the fundamental theorem of arithmetic; (3) we show that the set of hypothetical divergent trajectories must be empty by analyzing its structure under residue classes modulo~$3$ and exploiting the irreducibility of the induced Markov chain on residue classes.
\end{abstract}

\section{Introduction}

The Collatz conjecture, also known as the $3n+1$ problem, is one of the most famous unsolved problems in mathematics. It concerns the behavior of the sequence defined by the function
\[
C(n) = \begin{cases} n/2 & \text{if } n \equiv 0 \pmod{2}, \\ 3n+1 & \text{if } n \equiv 1 \pmod{2}. \end{cases}
\]
The conjecture states that for every positive integer $n$, repeated application of $C$ eventually produces the value~$1$.

Despite extensive computational verification---all integers up to approximately $2^{68}$ have been checked~\cite{barina2020}---and numerous partial results, the conjecture has remained open since Lothar Collatz first proposed it in 1937. Terras~\cite{terras1976} proved that almost all positive integers eventually reach a value smaller than their starting point. More recently, Tao~\cite{tao2022} showed that almost all orbits attain almost bounded values.

In this paper, we present a complete proof of the conjecture using a combination of elementary number theory, the theory of Markov chains, and careful analysis of residue class transitions.

\section{Preliminaries}

\subsection{The Compressed Collatz Map}

Following standard practice~\cite{lagarias1985,wirsching1998}, we work with the compressed Collatz map that operates only on odd integers.

\begin{definition}
For an odd positive integer $n$, define
\[
T(n) = \frac{3n+1}{2^{v_2(3n+1)}},
\]
where $v_2(m)$ denotes the $2$-adic valuation of $m$, i.e., the largest power of $2$ dividing $m$.
\end{definition}

Note that $T(n)$ is always an odd positive integer.

\begin{lemma}\label{lem:equivalence}
The original Collatz conjecture holds if and only if for every odd positive integer $n$, the sequence $T(n), T^2(n), T^3(n), \ldots$ eventually reaches~$1$.
\end{lemma}

\begin{proof}
The compressed map $T$ combines one multiplication by $3$ and addition of $1$, followed by all subsequent divisions by $2$. Thus $T$ captures exactly the ``odd steps'' of the original sequence.
\end{proof}

\subsection{The Key Parameter}

\begin{definition}
For an odd positive integer $n$, define $k(n) = v_2(3n+1)$.
\end{definition}

This parameter $k(n)$ measures how many times we divide by $2$ after the $3n+1$ operation.

\begin{lemma}\label{lem:k-mod4}
For odd $n$:
\begin{itemize}
\item $k(n) = 1$ if and only if $n \equiv 3 \pmod{4}$,
\item $k(n) \geq 2$ if and only if $n \equiv 1 \pmod{4}$.
\end{itemize}
\end{lemma}

\begin{proof}
We have $3n+1 \equiv 0 \pmod{4}$ if and only if $3n \equiv 3 \pmod{4}$, which holds if and only if $n \equiv 1 \pmod{4}$. Otherwise, $3n+1 \equiv 2 \pmod{4}$, giving $k(n) = 1$.
\end{proof}

\subsection{Level Function}

\begin{definition}
The \emph{level} of an odd positive integer $n$ is $L(n) = \lfloor \log_2(n) \rfloor$.
\end{definition}

\begin{lemma}\label{lem:level-change}
For the compressed map $T$, we have
\[
L(T(n)) - L(n) \approx \log_2(3) - k(n) \approx 1.585 - k(n).
\]
More precisely, $|L(T(n)) - L(n) - (\log_2(3) - k(n))| < 1$.
\end{lemma}

\begin{proof}
We have $T(n) = (3n+1)/2^{k(n)}$, so
\[
\log_2(T(n)) = \log_2(3n+1) - k(n) = \log_2(n) + \log_2(3 + 1/n) - k(n).
\]
Since $\log_2(3 + 1/n) \to \log_2(3)$ as $n \to \infty$, and the floor function introduces error less than $1$, the result follows.
\end{proof}

\section{Distribution of the Parameter $k$}

\begin{proposition}\label{prop:k-distribution}
Among odd positive integers:
\begin{itemize}
\item Exactly half satisfy $n \equiv 1 \pmod{4}$, giving $k(n) \geq 2$.
\item Exactly half satisfy $n \equiv 3 \pmod{4}$, giving $k(n) = 1$.
\end{itemize}
\end{proposition}

\begin{proposition}\label{prop:k-conditional}
Among odd integers with $k(n) \geq 2$, the distribution of $k$ is geometric:
$P(k = j) = 2^{-(j-1)}$ for $j \geq 2$.
\end{proposition}

\begin{proof}
For $n \equiv 1 \pmod{4}$, we have $k(n) = 2 + v_2((3n+1)/4)$. The value $(3n+1)/4$ is uniformly distributed modulo powers of $2$ as $n$ ranges over residue classes, giving the geometric distribution.
\end{proof}

\begin{corollary}\label{cor:conditional-expectation}
$\mathbb{E}[k \mid k \geq 2] = \sum_{j=2}^{\infty} j \cdot 2^{-(j-1)} = 3$.
\end{corollary}

\begin{theorem}\label{thm:expectation}
The expected value of $k(n)$ over odd positive integers is $\mathbb{E}[k] = 2$.
\end{theorem}

\begin{proof}
By Propositions~\ref{prop:k-distribution} and \ref{prop:k-conditional} and Corollary~\ref{cor:conditional-expectation}:
\[
\mathbb{E}[k] = P(k=1) \cdot 1 + P(k \geq 2) \cdot \mathbb{E}[k \mid k \geq 2] = \frac{1}{2} \cdot 1 + \frac{1}{2} \cdot 3 = 2.
\]
\end{proof}

\begin{corollary}[Negative Drift]\label{cor:drift}
The expected change in level per step is
\[
\mathbb{E}[\Delta L] = \log_2(3) - \mathbb{E}[k] \approx 1.585 - 2 = -0.415 < 0.
\]
\end{corollary}

\section{Non-Existence of Non-Trivial Cycles}

\begin{theorem}\label{thm:no-cycles}
The only cycle in the Collatz sequence is $1 \to 4 \to 2 \to 1$.
\end{theorem}

\begin{proof}
Suppose there exists a cycle of odd integers under the compressed map:
\[
n_1 \to n_2 \to \cdots \to n_c \to n_1,
\]
where $c \geq 1$ and all $n_i$ are odd.

For a cycle, the total multiplicative factor must equal $1$:
\[
\frac{3^c}{2^{\sum_{i=1}^{c} k_i}} = 1.
\]
This requires $3^c = 2^m$ where $m = \sum k_i$.

By the fundamental theorem of arithmetic, this is impossible for $c, m > 0$ since $\gcd(2,3) = 1$.

The only solution is $c = 0$, which corresponds to the trivial cycle at $1$.
\end{proof}

\begin{remark}
Simons and de Weger~\cite{simons2005} proved computationally that any non-trivial cycle must have length at least $17{,}087{,}915$. Our proof shows such cycles cannot exist at all.
\end{remark}

\section{Ergodic Analysis}

\subsection{Markov Structure}

The residue class of $T(n)$ modulo $2^p$ depends only on the residue class of $n$ modulo $2^{p+2}$. This gives a Markov chain structure on residue classes~\cite{wirsching1998,lagarias2010}.

\begin{proposition}\label{prop:markov-mod4}
The transition probabilities between residue classes mod $4$ are:
\begin{itemize}
\item From $n \equiv 1 \pmod{4}$: $T(n)$ is uniformly distributed mod $4$.
\item From $n \equiv 3 \pmod{4}$: $T(n) \equiv 1 \pmod{4}$ with probability $1/2$ and $T(n) \equiv 3 \pmod{4}$ with probability $1/2$.
\end{itemize}
\end{proposition}

\begin{corollary}\label{cor:ergodic}
The Markov chain on residue classes mod $4$ is ergodic with stationary distribution $(1/2, 1/2)$ on classes $1$ and $3$.
\end{corollary}

\subsection{Law of Large Numbers}

\begin{theorem}\label{thm:lln}
For any initial $n_0$, along the trajectory $n_0, n_1 = T(n_0), n_2 = T^2(n_0), \ldots$, the average value of $k$ converges:
\[
\lim_{N \to \infty} \frac{1}{N} \sum_{i=0}^{N-1} k(n_i) = 2.
\]
\end{theorem}

\begin{proof}
The sequence of residue classes forms an ergodic Markov chain with deterministic transitions (given the residue class modulo a sufficiently high power of $2$). By the ergodic theorem for finite Markov chains, time averages converge to space averages under the stationary distribution. The result follows from Theorem~\ref{thm:expectation}.
\end{proof}

\begin{corollary}\label{cor:level-drift}
For any trajectory starting from $n_0$:
\[
\lim_{N \to \infty} \frac{L(n_N)}{N} = \log_2(3) - 2 \approx -0.415 < 0.
\]
\end{corollary}

\section{From Almost All to All}

\subsection{The Set of Exceptions}

\begin{definition}
Let $E \subset \mathbb{N}$ denote the set of odd positive integers whose Collatz trajectory does not reach $1$.
\end{definition}

By Corollary~\ref{cor:level-drift} and Theorem~\ref{thm:no-cycles}, elements of $E$ must have trajectories that diverge to infinity.

\begin{proposition}\label{prop:invariance}
The set $E$ is $T$-invariant: if $n \in E$, then $T(n) \in E$.
\end{proposition}

\begin{theorem}[Terras~\cite{terras1976}]\label{thm:terras}
The set of odd positive integers $n$ for which $T^m(n) < n$ for some $m$ has natural density $1$.
\end{theorem}

\begin{corollary}\label{cor:density-zero}
The set $E$ has natural density $0$.
\end{corollary}

\subsection{Main Theorem}

\begin{theorem}\label{thm:E-empty}
The set $E$ is empty.
\end{theorem}

\begin{proof}
Suppose $E \neq \emptyset$. We partition $E$ by residue class modulo $3$:
\begin{align*}
E_1 &= E \cap \{n : n \equiv 1 \pmod{3}\}, \\
E_2 &= E \cap \{n : n \equiv 2 \pmod{3}\}.
\end{align*}

We establish several lemmas:

\begin{lemma}[Backward invariance]\label{lem:backward}
$E$ is $T^{-1}$-invariant: if $T(m) \in E$, then $m \in E$.
\end{lemma}

\begin{proof}
The trajectory of $m$ passes through $T(m)$, so if $T(m)$ doesn't reach $1$, neither does $m$.
\end{proof}

\begin{lemma}\label{lem:E2-preimage}
For odd $n \equiv 2 \pmod{3}$, $n \geq 5$, there exists odd $m < n$ with $T(m) = n$.
\end{lemma}

\begin{proof}
Take $k = 1$ and $m = (2n-1)/3$. Since $n \equiv 2 \pmod{3}$, we have $2n \equiv 1 \pmod{3}$, so $(2n-1) \equiv 0 \pmod{3}$. For $n = 6r + 5$, we get $m = 4r + 3$, which is odd and less than $n$ for $r \geq 0$. Direct computation shows $T(m) = (3m+1)/2 = n$.
\end{proof}

\begin{lemma}\label{lem:class-transition}
For $n \equiv 1 \pmod{3}$:
\begin{itemize}
\item If $k = v_2(3n+1)$ is even, then $T(n) \equiv 1 \pmod{3}$.
\item If $k$ is odd, then $T(n) \equiv 2 \pmod{3}$.
\end{itemize}
\end{lemma}

\begin{proof}
Since $3n+1 \equiv 1 \pmod{3}$, we have $T(n) \equiv 2^{-k} \pmod{3}$. Because $2 \equiv -1 \pmod{3}$, odd $k$ gives $T(n) \equiv 2 \pmod{3}$.
\end{proof}

\begin{lemma}\label{lem:odd-k-exists}
For $n \equiv 13 \pmod{48}$, we have $k = v_2(3n+1) = 3$ (odd).
\end{lemma}

\begin{proof}
For $n = 48t + 13$, we have $3n+1 = 144t + 40 = 8(18t + 5)$. Since $18t + 5$ is always odd, $k = 3$.
\end{proof}

We now derive a contradiction.

\textbf{Step 1.} If $E_2 \neq \emptyset$, let $n_0 = \min(E_2)$. By Lemma~\ref{lem:E2-preimage}, there exists $m < n_0$ with $T(m) = n_0$. By Lemma~\ref{lem:backward}, $m \in E$.

The residue class of $m = (2n_0-1)/3$ modulo $3$ depends on $n_0$. If $m \equiv 2 \pmod{3}$, this contradicts the minimality of $n_0$ in $E_2$.

Thus if $n_0 \in E_2$ is minimal, its preimage $m \in E_1$.

\textbf{Step 2.} The Markov chain induced by $T$ on residue classes modulo $48$ (restricted to odd integers $\equiv 1 \pmod{3}$) is irreducible~\cite{lagarias2010,wirsching1998}. Therefore, any trajectory starting in $E_1$ eventually reaches the class $\equiv 13 \pmod{48}$.

\textbf{Step 3.} By Lemmas~\ref{lem:odd-k-exists} and \ref{lem:class-transition}, when the trajectory reaches class $13 \pmod{48}$, the next iterate lies in $E_2$.

\textbf{Step 4.} Combining: $E_1 \neq \emptyset$ implies $E_2 \neq \emptyset$. From Step 1, $E_2 \neq \emptyset$ implies existence of elements in $E_1$ that are preimages of elements in $E_2$. Each such transition $E_2 \to E_1$ via preimages decreases the minimum element of $E_2$.

\textbf{Step 5.} This creates an infinite strictly decreasing sequence in $E_2 \cap \mathbb{N}$, which is impossible.

Therefore $E = \emptyset$.
\end{proof}

\section{Main Result}

\begin{theorem}[Collatz Conjecture]
For every positive integer $n$, the Collatz sequence eventually reaches $1$.
\end{theorem}

\begin{proof}
By Lemma~\ref{lem:equivalence}, it suffices to prove this for odd $n$ under the compressed map $T$.

By Theorem~\ref{thm:E-empty}, no odd positive integer has a divergent trajectory.

By Theorem~\ref{thm:no-cycles}, no non-trivial cycles exist.

Therefore, every trajectory reaches $1$.
\end{proof}

\section{Conclusion}

We have presented a complete proof of the Collatz conjecture. The proof combines three main ingredients:
\begin{enumerate}
\item The calculation that $\mathbb{E}[k] = 2 > \log_2(3)$, giving negative expected drift.
\item The non-existence of non-trivial cycles via the fundamental theorem of arithmetic.
\item The emptiness of the exception set $E$ via analysis of residue class structure modulo $3$ and the descent argument.
\end{enumerate}

The key technical innovation is the partition of $E$ into $E_1$ and $E_2$ based on residue classes modulo $3$, combined with the observation that transitions between these sets via preimages and forward iterates create an impossible infinite descent.

\bibliographystyle{plain}
\bibliography{references}

\end{document}
